\dosabstract{The causes and consequences of 3D-architectural changes at the onset of X-chromosome inactivation}{Alexandra Martitz, Edda Schulz}{<Lab missing>, Dpt. <Dpt missing>}{In the intricate realm of mammalian gene regulation, cis-regulatory elements play a crucial role in orchestrating precise gene expression during development. To exert their regulatory effect, they must come into physical contact with the promoters of their target genes, which is thought to be facilitated by loop extrusion. Here, ring-shaped cohesin complexes facilitate the contacts within defined regions of the genome, which are called topologically associating domains (TADs). To explore the intricate relationship between the cis-regulatory landscape, TAD structure, and gene expression, we employ the X inactivation centre (Xic) in mouse embryonic stem cells as a model system. The Xic regulates X-chromosome inactivation (XCI) in female mammals, in which one of the two X chromosomes is randomly chosen and silenced in early embryonic development to compensate for the double dosage of X-linked genes. Previous research has shown that the bipartite TAD structures at the Xic undergoes rewiring throughout development, exhibiting distinct patterns on the active and silenced X chromosomes. By using state-of-the-art techniques (i.e. Tiled-C, Tiled-MCC) to map the chromatin architecture at up to base-pair resolution, we have revealed nano-scale structures with unprecedent detail. Together with novel ChIPmentation data, we are able to show increased cohesin binding at activated sites as well as enhanced CTCF binding at rewired anchor sites. Together, our findings provide insights into the mechanisms of loop extrusion in regulating fine-scale chromatin architecture. }
\dosabstract{Genome-wide demographic analyses of balaenid whales revealed complex history of gene flow associated with past climate oscillation.}{Bai-Wei Lo, Francisca Martinez-Real, Stefan Mundlos}{<Lab missing>, Dpt. <Dpt missing>}{The balaenid whale, consisting of three species of right whales and the bowhead whale, represents an ancient and highly endangered lineage of marine mammals. In order to elucidate the evolutionary history of balaenid whales with respect to gene flow, a comprehensive analysis based on whole-genome data was conducted for all species within this group. Employing population genomic methodologies, we revealed the polytomic nature of extant right whales, identified passage of historical trans-equatorial migration, and provided estimates to the age of the group. Furthermore, we investigated the impact of glacial cycles on the connectivity of bowhead whale populations. Through the utilization of multiple complementary approaches to detect gene flow, we identified and characterized gene flow events from bowhead whales to North Atlantic right whales, offering detailed insights into the process. Lastly, we assessed the phenotypic consequences of interspecies gene flow. The outcomes of our study shed light on the intricate evolutionary history of modern balaenid whales, which have been profoundly shaped by ancient climate events.}
\dosabstract{Joint clustering and visualization of cells and genes for single-cell transcriptomics}{Clemens Kohl, Yan Zhao, Martin Vingron}{<Lab missing>, Dpt. <Dpt missing>}{Single-cell RNA-sequencing (scRNA-seq) allows researchers to study the heterogeneity of biological samples. Current methods however mainly focus on clustering the cells and attempt to determine the genes that define the clusters post-hoc through differential gene expression analysis and similar methods. This approach however often results in misleadingly low p-values and can find differences even in completely random data. Biclustering promises to solve this problem by simultaneously finding cell clusters and their respective marker genes. However, existing biclustering algorithms struggle with the higher noise and large size of scRNA-seq data and lack tools for a practical and concise visualization of results.  We here present two approaches on how Correspondence Analysis (CA) can be used to co-cluster cells and their respective marker genes in a single step, thereby solving the problems of current methods.  In order to tackle the task of identifying and annotating cell types in scRNA-seq data we developed Correspondence analysis based biclustering on Networks (CAbiNet). We build a graph connecting cells and genes in the noise and dimension reduced CA space by utilizing the Association Ratio to quantify the extent of how much the observed gene expression deviates from the expected expression in a cell. Clustering on this graph not only yields cell clusters and their cell type specific genes, but additionally allows us to visualize the cells and their defining marker genes in a single 2D plot that enables quick annotation and exploration of the obtained clusters. An additional advantage of this approach is that the graph could be extended to include additional data modalities such as scATAC-seq data.   Given the increasing size of scRNA-seq data, fast and highly scalable methods are required for analysis. We can furthermore exploit the highly directional geometry induced by CA to simultaneously cluster cells and their marker genes based on a total least squares regression approach that enables the analysis of even the largest datasets. }
\dosabstract{RiboSTS: a technology for combined single-cell sequencing of rRNA and mRNA}{Dmitrii Zagrebin, Matthew Kraushar}{<Lab missing>, Dpt. <Dpt missing>}{The ribosome is a multi-megadalton RNA-protein complex responsible for the translation of mRNAs into functional proteins, the final step of gene expression in every living cell. Ribosome biogenesis is one of the most complex and energy-demanding biological processes, which makes the ribosome number an excellent marker of a cell state. Studies performed in the Kraushar lab and by other groups have demonstrated that adjusting ribosome abundance is a universal mechanism for regulating gene expression, which is observed in various cellular processes including embryonic brain development, stem cell differentiation, and cancer. Nevertheless, little progress has been made recently in methods for measuring ribosome abundance and heterogeneity. In fact, all existing techniques exhibit one of the following flaws: they either measure bulk cell populations, or suffer from low throughput. RiboSTS (ribosome signatures targeted by high-throughput single-cell sequencing) is a new technology that overcomes both of these problems by sequencing rRNA in single cells in a high-throughput manner. In addition to rRNA, RiboSTS simultaneously captures the entire mRNA transcriptome for each individual cell. This dual sequencing approach allows RiboSTS to provide information about cell type and cell state signatures in a depth that has never been achieved before. Besides serving as a powerful tool for studying ribosome biogenesis and rRNA sequence variation on a single-cell level, RiboSTS will likely find use in clinical research for its ability to identify cells with upregulated ribosome number, such as cancer cells, or studying antibiotics resistance in prokaryotes which is usually caused by rRNA mutations. Making RiboSTS possible requires solving a series of challenges including cDNA synthesis from a highly structured RNA and careful balancing of sequencing reads between abundant rRNA and scarce mRNA, without compromising the quantification accuracy. Thus far, we performed a series of preliminary experiments, introduced innovations in the current single-cell RNA-seq protocol, and created a first draft of the RiboSTS.}
\dosabstract{MODELLING HUMAN HEPATO-BILIARY ORGANOGENESIS  FROM PLURIPOTENT STEM CELLS}{Irene Talon, Charlotte Grey-Wilson, Ludovic Vallier}{<Lab missing>, Dpt. <Dpt missing>}{During embryonic development, the liver, the pancreas and the extrahepatic biliary tree originate from the same region of the foregut endoderm. Although, studies in model organisms have provided insights into the early formation of these organs, the molecular mechanisms controlling the specification of their respective progenitors remain to be uncovered in human, due to the difficulty to obtain primary tissues at such early stages of development. To bypass this limitation, we have established a novel human pluripotent stem cell (hPSC)-derived foregut organoid system that recapitulates the formation of embryonic hepato-biliary-pancreatic organ buds. Our protocol allows the generation of self-organizing, three-dimensional human foregut organoids, containing progenitors for the liver (marked by AFP and ALB), the biliary three (KRT19 and SOX9) and the pancreas (PDX1). Importantly, the liver buds generated in vitro can further recapitulate liver organogenesis, by forming hepatoblast organoids (HBOs) which strongly resemble their in vivo counterparts. Accordingly, hPSC-derived HBOs can further differentiate into foetal hepatocytes or intrahepatic cholangiocytes, confirming their bipotentiality. Additionally, our foregut culture has the developmental potential to give rise to self-renewing biliary organoids, suggesting that our model system can also recapitulate, in part, the development of the biliary tree. We are now exploring the potential of our foregut organoids to generate pancreatic cells. Taken together, these results show that foregut progenitors can self-organize into organ buds and that this property can be further exploited to study early human organogenesis. This work will enhance our understanding on the mechanisms governing the cell fate choice during human hepato-biliary-pancreatic formation and will potentially serve as a new accessible platform for disease modeling and drug screening.}
\dosabstract{p53-dependent elimination of extraembryonic cells from the developing gut}{Julia Batki*, Sara Hetzel*, Dennis Schifferl, Adriano Bolondi, Lars Wittler, Bernhard G. Herrmann, Alexander Meissner }{<Lab missing>, Dpt. <Dpt missing>}{As the mouse embryo develops, cells segregate into embryonic and extraembryonic compartments that substantially differ both in their transcriptome and epigenome. The exception to this is the developing gut, a mosaic tissue jointly formed by embryonic and extraembryonic endoderm cells, which become transcriptionally highly similar at the onset of organogenesis. Yet, it has remained unknown whether these extraembryonic gut cells undergo similar conversion in their epigenetic state and if they are able to differentiate into more specialized endodermal cell types which persist into later development. Here we developed a stable two-color lineage tracing system to reliably distinguish and characterize cells of embryonic and extraembryonic origins throughout development. We find, despite the overall broad transcriptional plasticity of extraembryonic gut cells, that clear transcriptional signatures are preserved, and the extraembryonic DNA methylation landscape is retained as epigenetic memory of the lineage origin. Importantly, by midgestation, extraembryonic gut cells undergo programmed cell death and neighboring embryonic gut cells clear their remnants via non-professional phagocytosis. The elimination process requires the tumor suppressor p53, as gut cells of extraembryonic origin survive and further differentiate in the absence of p53 both in vivo and in vitro. Our study elucidates a selective cell clearance mechanism during tissue morphogenesis and provides a novel role of p53 in organogenesis in a mammalian system.}
\dosabstract{Controlling the composition and function of biomolecular condensates using a micropeptide “killswitch”}{Yaotian Zhang, Henri Niskanen, Ida Stöppelkamp, Denes Hnisz}{<Lab missing>, Dpt. <Dpt missing>}{In recent years, the condensation behavior of biomolecules has emerged as a novel regulatory modality for shaping fundamental cellular processes, such as gene transcription, silencing, and chromatin remodeling. While significant insights have been gained through perturbation assays that dissect individual components of these biomolecular condensates, it is crucial to acknowledge their inherent complexity. Condensates are intricate assemblies, encompassing a diverse array of biomolecules, including proteins, RNA, and DNA, each contributing to the collective functionality of the condensate. Isolating individual components may provide only a partial glimpse into the inner workings of these condensates. To obtain a more comprehensive understanding of biomolecular condensates, it is imperative to investigate their emergent properties, such as the material states and how these properties facilitate their vital biological functions. Currently, tools for perturbing the material properties of condensates are limited. Therefore, we introduce a novel micro-peptide termed "Killswitch (KS)" designed to solidify condensates that enables us to understand condensate biology better and potentially progress to condensate-specific therapeutics. }
